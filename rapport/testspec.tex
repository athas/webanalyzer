\documentclass[a4paper,oneside,article, titlepage]{memoir}
\usepackage[T1]{fontenc}
\usepackage{textcomp}
\usepackage[utf8]{inputenc}
\usepackage[danish]{babel}
\usepackage[garamond]{mathdesign}
\usepackage{url}
\usepackage{graphicx}

\DeclareTextFontCommand{\textfleur}
{\fontencoding{T1}\fontfamily{FleurCornerCaps}\selectfont}


\renewcommand{\ttdefault}{pcr} % bedre typewriter font
\renewcommand{\rmdefault}{ugm} % garamond


\usepackage{lettrine}

%\overfullrule=5pt

%\setsecnumdepth{part}

\title{Testspecifikation  \\ \small{Førsteårsprojekt}}

\author
{
  Gruppe 1:\\
  Troels Henriksen (athas@sigkill.dk)\\
  Jesper Reenberg (reenberg@kampsax.dtu.dk)\\
  Martin Dybdal (dybber@dybber.dk)\\ \\
  Vejledere: Dina og Kasper
}


%\setcounter{tocdepth}{3}
%\setcounter{secnumdepth}{2}

\pagestyle{plain}

\date{\today}

\begin{document}
\maketitle
\tableofcontents*
\newpage



% Testspecifikationen SKAL beskrive:

% * planlægning af testen, herunder jeres mål med testen og hvad I
% finder særligt kritisk for jeres projekt at få testet;

% * resultaterne af testen, herunder hvilke fejl er udbedret og hvilke
% fejl findes stadig i programmet;

% * en test af funktionalitet, herunder hvordan I har genereret
% testeksempler, indholdet af testeksemplerne (fortløbende
% nummereret), resultaterne fra kørsel af testeksempler;

% * en test af brugsvenlighed, herunder en begrundelse for valg af
% evalueringsteknik, en liste over de brugsvenlighedsproblemer som er
% identificeret (fortløbende nummereret), en beskrivelses af hvordan
% de kan løses.

% Bedømmelsen af testspecifikationen fokuserer på at I: (a) har
% besvaret alle fire punkter ovenfor, (b) har planlagt testen med
% hensyntagen til jeres projekts fokus, (c) ved jeres test dækker
% samtlige krav, dvs. har mindst et testeksempel for hvert krav, (d)
% har genereret testeksempler systematisk, (e) har beskrevet løsninger
% på væsentlige funktionalitets- og brugsvenlighedsproblemer, (f) har
% lavet jeres brugsvenlighedsevaluering baseret på realistiske
% opgaver, (g) har fundet fejl ? både i funktionstesten og ved
% evalueringen af brugsvenlighed ? eller har en overbevisende
% forklaring på hvorfor ingen fejl blev fundet.

% Upload filen i jeres gruppes folder. Testspecificationen skal
% afleveres som ét dokument og oploades i pdf-format (navn
% "Testspecifikation") senest den 6. juni 2007.
 

\chapter{Strategi}
Vores test er delt i to. Den første del --- funktionstesten --- skal
afgøre om programmet kan alle de ting vi påstår det kan og at alle
disse funktioner virker korrekt. Det optimale ville være hvis denne
test kunne automatiseres, men det er ikke tilfældet og vi må derfor
teste det manuelt.

Den anden del af testen --- brugertesten --- skal bruges til at finde
ud af om brugervejledningen er forståelig og om brugeren kan finde ud
af at bruge programmet ved at læse brugermanualen. Derudover skal
testen bruges til at finde ud af om resultatsiderne er vanskelige at
læse. 

\chapter{Funktionstest}

\section{Dybde angivelse}
Hvis man angiver en maksimal dybde af analysen, så skal programmet
respektere denne og gøre det korrekt.

\subsection{Ækvivalens-klasser}
\begin{itemize}
\item 
\end{itemize}

\section{Links finder}
Programmet skal finde alle interne links på hjemmesiden, som
eksiterer.


\section{Analyse indhold}
Al \underline{relevant} tekst fra hjemmesiden skal fremgå af
analyseresultatet, herunder hjemmesidens titel.

\section{Stavekontrol}
Hvis en side er angivet til dansk og en dansk ordbog til aspell er
installeret, så skal programmet kunne tjekke for stavefejl.
\subsection{Ækvivalens-klasser}
\begin{itemize}
\item aspell er ikke installeret --- Stavekontrol skal ikke udføres og
  brugeren skal have besked.
\item De korrekte sprogpakker er ikke installeret --- Stavekontrol
  skal ikke udføres på de relevante sektioner og brugeren skal have
  besked.
\item 
\end{itemize}

\section{Respekter robots.txt}
Programmet skal tjekke det angivne websteds robots.txt og sørge for at
websider angivet i filen ikke bliver hentet og analyseret.

\section{Udfør analyser angivet af brugeren}


\section{Indkodning}
Programmet skal virke på hjemmesider indkodet med ASCII
tegnsættet. Specielt skal det testes at de dansk tegn (æøåÆØÅ) virker
korrekt.

\section{Filtrering mekanisme}
Programmet skal fra sortere


\section{Output-mappe}
Hvis man angiver en output-mappe

\chapter{Brugertest}


\end{document}
