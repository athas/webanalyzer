\documentclass[a4paper,oneside,article, titlepage]{memoir}
\usepackage[T1]{fontenc}
\usepackage{textcomp}
\usepackage[utf8]{inputenc}
\usepackage[danish]{babel}
\usepackage[garamond]{mathdesign}
\usepackage{url}
\usepackage{graphicx}

\DeclareTextFontCommand{\textfleur}
{\fontencoding{T1}\fontfamily{FleurCornerCaps}\selectfont}


\renewcommand{\ttdefault}{pcr} % bedre typewriter font
\renewcommand{\rmdefault}{ugm} % garamond


\usepackage{lettrine}

%\overfullrule=5pt

%\setsecnumdepth{part}

\title{Testspecifikation  \\ \small{Førsteårsprojekt}}

\author
{
  Gruppe 1:\\
  Troels Henriksen (athas@sigkill.dk)\\
  Jesper Reenberg (reenberg@kampsax.dtu.dk)\\
  Martin Dybdal (dybber@dybber.dk)\\ \\
  Vejledere: Dina og Kasper
}


%\setcounter{tocdepth}{3}
%\setcounter{secnumdepth}{2}

\pagestyle{plain}

\date{\today}

\begin{document}
\maketitle
\tableofcontents*
\newpage



% Testspecifikationen SKAL beskrive:

% * planlægning af testen, herunder jeres mål med testen og hvad I
% finder særligt kritisk for jeres projekt at få testet;

% * resultaterne af testen, herunder hvilke fejl er udbedret og hvilke
% fejl findes stadig i programmet;

% * en test af funktionalitet, herunder hvordan I har genereret
% testeksempler, indholdet af testeksemplerne (fortløbende
% nummereret), resultaterne fra kørsel af testeksempler;

% * en test af brugsvenlighed, herunder en begrundelse for valg af
% evalueringsteknik, en liste over de brugsvenlighedsproblemer som er
% identificeret (fortløbende nummereret), en beskrivelses af hvordan
% de kan løses.

% Bedømmelsen af testspecifikationen fokuserer på at I: (a) har
% besvaret alle fire punkter ovenfor, (b) har planlagt testen med
% hensyntagen til jeres projekts fokus, (c) ved jeres test dækker
% samtlige krav, dvs. har mindst et testeksempel for hvert krav, (d)
% har genereret testeksempler systematisk, (e) har beskrevet løsninger
% på væsentlige funktionalitets- og brugsvenlighedsproblemer, (f) har
% lavet jeres brugsvenlighedsevaluering baseret på realistiske
% opgaver, (g) har fundet fejl ? både i funktionstesten og ved
% evalueringen af brugsvenlighed ? eller har en overbevisende
% forklaring på hvorfor ingen fejl blev fundet.

% Upload filen i jeres gruppes folder. Testspecificationen skal
% afleveres som ét dokument og oploades i pdf-format (navn
% "Testspecifikation") senest den 6. juni 2007.
 

\chapter{Testplan}

\chapter{Funktionstest}

\section{}

\chapter{Brugertest}
\end{document}
