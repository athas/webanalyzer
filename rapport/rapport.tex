\documentclass[a4paper,oneside,article]{memoir}
\usepackage[T1]{fontenc}
\usepackage{textcomp}
\usepackage[utf8]{inputenc}
\usepackage[danish]{babel}
\usepackage[garamond]{mathdesign}
\usepackage{url}
\usepackage{graphicx}
\usepackage{pdflscape}
\usepackage{longtable}

\DeclareTextFontCommand{\textfleur}
{\fontencoding{T1}\fontfamily{FleurCornerCaps}\selectfont}


\renewcommand{\ttdefault}{pcr} % bedre typewriter font
\renewcommand{\rmdefault}{ugm} % garamond

\usepackage{lettrine}

%\overfullrule=5pt

%\setsecnumdepth{part}

\title{Hjemmesideanalyse  \\ \small{Førsteårsprojekt}}

\author
{
  Gruppe 1:\\
  Troels Henriksen (athas@sigkill.dk)\\
  Jesper Reenberg (reenberg@kampsax.dtu.dk)\\
  Martin Dybdal (dybber@dybber.dk)\\ \\
  Vejledere: Dina og Kasper
}


\setcounter{tocdepth}{2}
\setcounter{secnumdepth}{2}

\pagestyle{plain}

\date{\today}

\begin{document}
\maketitle
\newpage
\tableofcontents*
\newpage

\chapter{Indledning}
\label{indledning}
Mange hjemmesider forfattes i dag uden omtanke for om sidens målgruppe
er i stand til at læse sidens indhold --- det sproglige niveau er
ganske enkelt for højt. Ydermere forfattes indholdet af mange
hjemmesider med primitive værktøjer (f.eks. simple tekstfelter direkte
på siden, eller simple skriveprogrammer), der mangler de
hjælpefeatures som findes i almindelige skriveprogrammer. Dette
resulterer i at hjemmesider ofte indeholder flere stavefejl end
gennemsnitlig tekst. Disse mangler kan gøre hjemmesider sværere at
bruge for den tilsigtede målgruppe, og derved reducere deres
effektivitet, ligegyldigt hvad målet med siden så end er.

Baseret på denne problemstilling har vi implementeret et program
designet til at bistå hjemmesideforfattere med at forbedre den
sproglige kvalitet af deres sider, hvad enten dette er ved at reducere
antallet at stavefejl eller at omskrive unødigt kompliceret tekst.

Denne rapport beskriver design- og implementeringsprocessen bag vores
løsning af problemet, en arbejdsproces der blev udført i forbindelse
med Førsteårsprojektet på DIKU i blok 4, 2007.

Det forventes at læseren af denne rapport har en hvis grad af teknisk
kompetence, har erfaring med programmering og har en overordnet
forståelse for sproget HTML\footnote{Et sprog der bruges til at
  beskrive struktur og indhold af websites}.

\chapter{Analyse}

I dette afsnit vil vi beskrive de tanker vi gjorde os omkring
problemet, og de forskellige løsningsmuligheder vi vurderede.

\section{Målgruppe}
\label{målgruppe}
Der er adskillige mulige målgrupper en løsning på problemet kunne være
orienteret imod --- fra hjemmebrugere der skriver personlige sider,
over de individuelle skribenter på et større website til selve
webmasteren på en stor side. Ydermere kan et websites indhold
fortolkes på flere forskellige måder --- enten som almindelig tekst
omgivet af HTML-data uden betydning, eller med et forsøg på at benytte
den information som HTML-strukturen udtrykker. Sidstnævnte gør det
væsentligt lettere for et program at undersøge indholdet på et
website, idet mere information om sidernes indhold så er eksplicit
udtrykket i sidernes kode, i stedet for at programmet skal
``gætte''. Dette har også relevans i forhold til målgruppen, idet
HTML's avancerede features for at udtrykke semantisk information
omkring teksten (information om sprog, forkortelser, osv) primært
benyttes af tekniske brugere, og det virkede paradoksalt at lave en
løsning til ikke-tekniske brugere der virker bedst med data produceret
af tekniske brugere. Vi besluttede os for at det ville være et spild
at undlade at udnytte HTML's muligheder for at angive semantisk
information omkring teksten, men da en så lille del af verdens
websites benytter sig af HTML's muligheder indenfor dette område,
ville det heller ikke være hensigtsmæssigt at handikappe programmet
til kun at virke acceptabelt på denne form for HTML. Vores endelige
målgruppe endte med at blive todelt --- webmastere og
(semi-)professionelle skribenter. Dette valg baserer vi på at både
webmastere og skribenter har en stor interesse i at deres sider eller
tekst er let at læse, og har motivation til at sætte sig ind i
hvorledes man skriver god HTML, og derved gør det muligt for et
program at foretage en så god undersøgelse af websitet som
muligt.\footnote{Det kan nævnes at HTML-sider rig på semantisk
  information også gør det lettere for brugere af alternative
  weblæsere, som f.eks. blinde der benytter sig at netoplæsere, at
  benytte sig af siden} De, der ikke har interesse eller motivation
til at sætte sig ind i teknikken, har sandsynligvis også mindre
interesse i at deres websites er lette at læse --- ejerne af
professionelt orienterede websites har en væsentligt større interesse
i at deres sider er let læsbare end forfatterne af f.eks. en families
hjemmeside.

\section{Overvejelser omkring brugsscenarier}
\label{brugsovervejelser}
Vi blev hurtigt enige om at det er langt mere interessant at analysere
et website som helhed, end kun at analysere en specifik HTML-side, og
samtidigt at det er en nødvendighed at løsningen kan gennemgå
(``crawle'') et online website, og ikke være afhængig af at det i
forvejen er hentet og er beliggende som filer på brugerens
harddisk. Hvor det er relativt nemt at få et program, der kan
analysere et helt website, til kun at analysere en enkelt side, er det
omvendte langt sværere, idet man så skal implementere crawling seperat
fra programmet. Ydermere er især den ene del af vores målgruppe ---
webmastere --- sandsynligvis langt mere interesserede i at
analysere hele det website de er ansvarlige for, for at finde ud af
hvilke undersider der er uacceptabelt svære at læse, end at analysere
enkeltsider. Idet vi kan let kan forestille os at det kan tage lang
tid at analysere alle undersider på et stort website, forestiller vi
os også et brugsmønster hvori programmet automatisk bliver kørt med et
fast interval, og producerer et resultat som er tilgængeligt for alle
der arbejder på siden. Dette er langt mere effektivt end at alle
skribenter hver især skal køre tidskrævende analyser hver for sig. Dog
opfatter vi det også som sandsynligt at skribenter, der arbejder på at
forbedre læsbarheden af en specifik underside, vil have en mulighed
for hurtigt at analysere denne side for at se resultatet af deres
ændringer, i stedet for at de skal vente på at den automatiske analyse
bliver kørt (og bliver færdig igen).

\section{Brugergrænseflade og resultatpræsentation}
\label{brugergrænseflade}
En løsning på opgavens overordnede problem kan formuleres som at
analysere et website og producere en eller anden form for resultat,
der indeholder information om dets læsbarhed. Der er flere forskellige
muligheder for hvorledes brugeren skal kunne sætte en analyse i gang,
og hvorledes brugeren skal præsenteres for resultatet, og det rette
valg afhænger primært af den valgte målgruppe. Valget af
resultatpræsentation har stor betydning for løsningen, idet både valg
af teknologi, udformning af design og strategien for den endelige
implementation bliver påvirket hvorledes brugeren skal præsenteres for
resultatet. Der er også forskel på hvor godt forskellige
brugergrænseflader gør det muligt at analysere hele websites,
automatisere kørsel af programmet eller dele resultatet med andre
brugere (så hver bruger ikke skal køre en ny analyse fra start
af). Forskellige muligheder blev overvejet:

\begin{description}
\item[Integreret med webbrowser:]
  Programmet kan være en del af brugerens webbrowser --- når
  programmet er aktivt vil det, når brugeren besøger en side, indikere
  læsbarheden af siden. Dette kan ske på forskellige måder --- når
  brugeren holder musen over en sætning eller et afsnit kan f.eks. et
  LIX-tal (eller lignende) vises, et panel kan komme til syne i siden
  af browseren der indeholder information om sidens læsbarhed, eller
  teksten på siden kan farves efter hvor svær den er at læse. Fordelen
  ved denne løsning er at den er en udvidelse af brugerens normale
  brug af Internettet, og at information om læsbarheden af en given
  side præsenteres sammen med selve siden, og derved gør det let at
  forbinde læsbarhedsresultatet med sidens indhold. Det føles
  naturligt og intuitivt at bevæge sig ned af en side og lede efter
  indikerede problemer. 

  Ulempen ved denne præsentationsform er at det ikke umiddelbart lader
  sig gøre at analysere et helt website og kun søge detaljeret
  information om de mindst læsbare sider. En anden ulempe er at
  integrationen med browseren kan gøre det svært at automatisere
  kørslen af programmet, og at en præsentation der foregår i
  forbindelse med browserens egen visning af siden kan gøre det svært,
  for ikke at sige umuligt, at gøre resultatet tilgængeligt for andre
  brugere.

\item[Webapplikation:]
  Programmets brugerinterface kan være udformet som et website (en
  webapplikation), hvor brugeren kan angive adressen på et andet
  website, som så bliver analyseret. Præsentationen af resultatet
  kunne være en dynamisk genereret kopi af websitet, som brugeren kan
  navigere igennem online som hvis det var det oprindelige website,
  men hvor der på hver side er indsat information om tekstens
  læsbarhed. 

  Problemet med denne mulighed er at en komplet analyse af et
  middelstort website sandsynligvis tager længere tid end de fleste
  brugere er villige til at vente på en hjemmeside i, og at der ikke
  er en umiddelbar måde at dele resultatet med andre brugere på.

\item[Grafisk standalone-applikation:] 
  Præsentationen kan udføres omtrent som nævnt i første mulighed, men
  i stedet for at køre som en del af webbrowseren, er det et vindue
  for sig selv, og det er selv ansvarligt for at vise indholdet af en
  analyseret side og sætte det i forbindelse med
  analyseresultatet. Hvis brugeren ønsker at analysere et helt website
  kan dette også gøres ved at programmet systematisk følger links på
  websitet og analyserer hver side det finder. Der kan så produceres
  en liste over de analyserede sider, hvorfra brugeren kan vælge at få
  mere detaljeret information (f.eks. den førnævnte visning af siden
  med uacceptabelt svært læsbar tekst markeret). Denne liste kan
  eventuelt sorteres efter sidernes læsbarhed eller lignende.

  Denne løsnings primære problem er at det er relativt svært at
  automatisere grafiske programmer, og at det stadigvæk ikke
  umiddelbart er muligt at dele et analyseresultat med flere
  brugere. En løsning på dette problem kunne være at opdele programmet
  i to dele --- en del der udfører den konkrete analyse og laver
  resultatet som en fil eller lignende, og en del der visuelt kan vise
  resultatet ud fra denne fil. En variant over dette er hvad næste
  mulighed går ud på.

\item[Output i form af HTML:]
  Programmet kan som foreslået ovenfor være konceptuelt opdelt i
  to --- en del der ``crawler'' et website og producerer et
  analyseresultat, og en del der giver en visuel præsentation af
  resultatet for brugeren. Dog er finessen at den første del
  præsenterer et sæt af HTML-filer, som kan læses af enhver
  webbrowser, hvilket betyder at vi ikke har behov for at implementere
  præsentationsdelen af løsningen. Programmet skal blot producere
  HTML-filer indeholdende websitets oprindelige tekst i forbindelse
  med analyseresultatet, og gøre det muligt at se alle undersider på
  websitet (f.eks. ved at producere en HTML-fil der indeholder en
  liste over alle de sider programmet besøgte under kørslen). Idet
  programmet ikke har behov for selv at præsentere resultatet visuelt
  (det håndteres af brugerens webbrowser) kan det implementeres som et
  kommandolinjeprogram, hvilket gør det nemt at automatisere, og da
  programmets output er almindelige HTML-filer kan disse nemt gøres
  tilgængelige for andre brugere.
\end{description}

Vi valgte sidstnævnte mulighed, og valgte også at implementere
løsningen som et kommandolinjeprogram. Dette sikrer et af vores
brugsscenarier, hvori programmet køres med et regulært interval og
producerer resultater der er tilgængelige for alle der arbejder på
websitet, og samtidigt er vi overbeviste om at det er muligt at lave
et så simpelt kommandolinjeinterface at skribenter hurtigt også kan
lære at bruge det. At output er i form af HTML-filer gør det også
muligt at lave nogle senere udvidelser eller alternative
brugergrænseflader --- en interessant idé kunne være en webapplikation
hvori brugere kan ``bestille'' analyser, og få en email med et link
til den producerede analyse (programmets output i form af HTML) når
den er færdig.

\section{Konkrete features}

Idet vores primære brugsscenarie omhandler analyse af alle sider på et
helt website tilgængeligt online (se sektion \ref{brugsovervejelser}),
er vores program nødt til at være i stand til at tage forbindelse
forbindelse til en webserver, hente en HTML-fil, forstå indholdet af
HTML-filen, udføre en analyse af teksten i filen og følge hyperlinks
angivet i HTML-filen til andre sider på websitet. At følge hyperlinks
fra side til side er vores eneste mulighed for at besøge alle sider på
websitet, hvilket giver mulighed for at der findes isolerede sider som
vi ikke er i stand til at besøge, men da menneskelige besøgende på
siden også er nødt til at følge hyperlinks for at komme omkring, og
det derfor er yderst sjældent at der findes isolerede sider på et
website, mener vi ikke at dette er et problem.

For at analysere teksten på siderne valgte vi at benytte to
analysemetoder -
læsbarhedsindeks\footnote{\url{http://www.elkan.dk/lixtal.asp}} og
FKRT\footnote{\url{http://en.wikipedia.org/wiki/Flesch-Kincaid_Readability_Test}}.
Læsbarhedsindeks (herefter refereret til som LIX) er en klassisk
algoritme til at analysere læsbarheden af en tekst, den er velforstået
af mange og virker derfor som et godt valg til programmet. Ydermere
har vi også valgt at benytte os af FKRT, der primært bruges til
engelsk tekst, men som også producerer brugbare resultater for dansk
tekst. Desuden fandt vi det også relevant at foretage stavekontrol af
de analyserede sider, idet mange sider bliver skrevet med primitive
værktøjer der ikke indeholder automatisk stavekontrol (se sektion
\ref{indledning}). Vi har nævnt at det er oplagt at benytte os af den
information som en sides HTML fortæller om dens tekst (se sektion
\ref{målgruppe}), og det er i forbindelse med stavekontrollen især
nærliggende at inddrage HTML's \texttt{lang}-attribut, som angiver en
teksts sprog, således at vi altid bruger den rigtige ordbog til at
stavekontrollen. Det gør det muligt for vores program at fungere med
sider der indeholder tekst skrevet på flere sprog, uden at det
ukorrekt angiver forkert stavede ord. Slutteligt taler vores egen
erfaring for at en hændelig skrivefejl er fejlagtig gentagelse af ord,
så denne fejl valgte vi ligeledes gøre opmærksom på (skriveprogrammer
med grammatikcheck fanger denne fejl, men det gør en selvstændig
stavekontrol, som den i vores program, ikke).

Som nævnt i sektion \ref{brugergrænseflade} består programmets output
af HTML-filer --- disse filer skal kommunikere resultatet på en
forståelig måde, idet de skal kunne stå alene når de ses i en
browser. Vi fandt det mest intuitivt at programmet laver en fil med
analyseresultater for hver side det besøger, og slutteligt laver en
overordnet indeks-fil der indeholder en liste over alle de besøgte
sider, og indeholder et link til den fil, der indeholder detaljerne
omkring analyse af siden. Disse sider indeholder den relevante sides
tekst, samt talværdier der angiver sidens læsbarhed (LIX-tal, FRE-tal
og FKGL-tal). Teksten på siden er opdelt i afsnit\footnote{I vores
  program dækker begrebet afsnit både over HTML's koncept om afsnit,
  adskilt af \texttt{p}-tags, samt over anden selvstændig tekst, som
  er strukturelt separat i HTML-dokumentet}, og for hvert afsnit
optræder der igen information om afsnittets læsbarhed --- dette er for
at gøre det lettere at finde, og få rettet, de mindst læsbare afsnit
på en side. Hver enkelt sætning har også en værdi associeret med sig,
der angiver hvor læsbar sætningen er - men eksperimenter viste at det
var uholdbart at præsentere sætninger med tre tilknyttede talværdier,
på samme måde som med afsnittene, idet det ødelagde flowet i teksten,
og gjorde det trægt og besværligt at læse analysesiderne
igennem. Derfor bliver læsbarheden af sætningerne i et afsnit i stedet
indikeret via deres baggrundsfarve, som går fra lysegrøn for meget
læsbare sætninger, til blodrød for ulæselige sætninger. Vores
oplevelse er at farverne hurtigt gør det muligt at skimme en
analyseside og identificere afsnit der ser ``meget røde'' ud, uden at
man har behov for at gennemgå alle afsnittenes analyseresultater for
at finde den med den dårligste værdi.

Programmet understøtter flere features, især omkring
indstillingsparametre og håndtering af specifikke HTML-tags, men disse
er ikke analysemæssigt særligt interessante, og vi henviser derfor til
den vedlagte kravspecifikation for en mere detaljeret gennemgang.

\chapter{Programdesign}

\chapter{Implementation}

\chapter{Afprøvning}

\chapter{Konklusion}

\end{document}
