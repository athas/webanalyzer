\documentclass[a4paper,10pt,draft]{article}

\usepackage[utf8x]{inputenc}
\usepackage[danish]{babel}

\begin{document}

\title{Møde om endelig kravspecifikation}
\author{Tilstedeværende: Martin Dybdal, Troels Henriksen og Dina.\\
\textit{Fraværende: Jesper Reenberg}\\
\textit{Ordstyrer: Martin Dybdal}.\\
\textit{Referent: Troels Henriksen}}
\maketitle


\textbf{På mødet blev følgende ting diskuteret:}
\\

\begin{itemize}

    \item Der skal tilføjes flere overskifter til afsnitene. Specielt analysen.

	\item Vi kan evt rende ind i problemer med hensyn til ``code review' fra andre grupper grunet vores valg af programmerings sprog, men det blev aftalt at dette ikke skulle være en hindring for os.

	\item Dina påpegede at der mangler at blive defineret hvad ``sidens tekst'' er.

	\item Der blev diskuteret hvorvidt vores kravspec passede ind i ``skabelonen'' da eksempler var tynde og begrundelser ofte var gentalgeser. Men Dina kommenterede : ``Det spiller ingen rolle på så lille skala som det foregår på.''

	Heraf at hvert afsnit i analysen skal indeholde en problemstilling og en løsning.

	\item Krav 2 blev diskuteret og der skal beskrives entydigt at vi med en analyse af en undersdie, mener at foretage analyser med de nedenfor beskrevne algoritmer og knytte deres resultagter sammen med siden som output (dog strippet for eventuel formatering / billeder som kunne forvirre det overordnede billede),

	\item Dina påpeger at vi generelt skal læse vore ting mere igennem.

	\item Sidesværhedsgraden skal defineres som værende en værdi baseret på reultatet af LIX- of FKRT-algoritmen, men da disse først beskrives senere er der behov for referencer.

	\item  Afsnittet der beskriver krav til målgruppe og læsere skal pilles ud for seg og have en individuel overskrift.

	\item Dina påpeger at det indledende afsnit i analysen kunne undværes da dette var for meget ``hyggesnak''.

\end{itemize}

\end{document}
